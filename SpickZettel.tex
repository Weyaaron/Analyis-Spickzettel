%%%%%%%%%%%%%%%%%%%%%%%%%%%%%%%%%%%%%%%%%
% Cheatsheet
% LaTeX Template
% Version 1.0 (12/12/15)
%
% This template has been downloaded from:
% http://www.LaTeXTemplates.com
%
% Original author:
% Michael Müller (https://github.com/cmichi/latex-template-collection) with
% extensive modifications by Vel (vel@LaTeXTemplates.com)
%
% License:
% The MIT License (see included LICENSE file)
%
%%%%%%%%%%%%%%%%%%%%%%%%%%%%%%%%%%%%%%%%%

%----------------------------------------------------------------------------------------
%	PACKAGES AND OTHER DOCUMENT CONFIGURATIONS
%----------------------------------------------------------------------------------------

\documentclass[11pt]{scrartcl} % 11pt font size

\usepackage[ngerman]{babel}
\usepackage[utf8]{inputenc} % Required for inputting international characters
\usepackage[T1]{fontenc} % Output font encoding for international characters

\usepackage[margin=0pt, landscape]{geometry} % Page margins and orientation

\usepackage{graphicx} % Required for including images

\usepackage{xcolor} % Required for color customization
\definecolor{mygray}{gray}{.75} % Custom color

\usepackage{url} % Required for the \url command to easily display URLs
\usepackage{tikz} % Required for the \keystroke command defined below to easily display Keys to press

\usepackage[ % This block contains information used to annotate the PDF
colorlinks=false, 
pdftitle={Cheatsheet}, 
pdfauthor={Aaron Wey, Markus Richter}, 
pdfsubject={Compilation of useful shortcuts}, 
pdfkeywords={Bash, Cheatsheet}
]{hyperref}

\setlength{\unitlength}{1mm} % Set the length that numerical units are measured in
\setlength{\parindent}{0pt} % Stop paragraph indentation

\renewcommand{\dots}{\ \dotfill{}\ } % Fills in the right amount of dots

\newcommand{\command}[2]{#1~\dotfill{}~#2\\} % Custom command for adding a shorcut

\newcommand{\sectiontitle}[1]{\paragraph{#1} \ \\} % Custom command for subsection titles


\colorlet{cmdColor}{blue}
\colorlet{argColor}{red}


\colorlet{cmdColor}{blue}
\colorlet{argColor}{red}

\newcommand{\Cmd}[1]{
\color{cmdColor}\$ #1
\color{black}
}

\newcommand{\Arg}[1]{
\color{argColor}#1 \color{black}
}


\newcommand{\Cmdp}[1]{
\color{cmdColor} #1
\color{black}
}
\usetikzlibrary{arrows, shapes, positioning, shadows}

\newcommand*\keystroke[1]{%
	\tikz[baseline=(key.base)]
	\node[%
	draw,
	fill=white,
	drop shadow={shadow xshift=0.25ex,shadow yshift=-0.25ex,fill=black,opacity=0.75},
	rectangle,
	rounded corners=2pt,
	inner sep=1pt,
	inner xsep=2pt,
	line width=0.5pt,
	font=\scriptsize\sffamily
	](key) {#1\strut}
	;
}

%----------------------------------------------------------------------------------------

\begin{document}

\begin{picture}(297,210) % Create a container for the page content

%----------------------------------------------------------------------------------------
%	TITLE SECTION 
%----------------------------------------------------------------------------------------

\put(10,200){ % Position on the page to put the title
\begin{minipage}[t]{210mm} % The size and alignment of the title
\section*{Cheatsheet Termumformungen Analysis für InformatikerInnen} % Title
Die Formeln sind grob nach Nützlichkeit sortiert.
\end{minipage}
}

%----------------------------------------------------------------------------------------
%	FIRST COLUMN SPECIFICATION
%----------------------------------------------------------------------------------------

\put(10,180){ % Divide the page
\begin{minipage}[t]{85mm} % Create a box to house text


    \sectiontitle{Brüche} \command{ \[  \frac{a}{b} \; + \; \frac{c}{d} \; = \;
    \frac{a \cdot d + c \cdot b}{b \cdot d}  \]}{}
    \command{ \[  \frac{a}{b} \; \cdot \; \frac{c}{d} \; = \; \frac{a \cdot c}{b \cdot d} \]}{}
    \command{ \[ \frac{a}{b}  :  \frac{c}{d} \; = \; \frac{a}{b} \; \cdot \; \frac{d}{c} \; = \; \frac{a \cdot d}{b \cdot c} \]}{}
    \command{ \[ {\frac  {a}{b}}:n\;=\;{\frac  {a}{b\cdot n}}\quad  \] }{}
    \command{ \[ \quad n:{\frac  {a}{b}}\;=\;{\frac  {n\cdot b}{a}} \] }{} 
    \command{ \[  \frac{a^n}{b^m} = a^n \cdot b^{-m}\] }{} 
    \command{ \[  \frac{a^n}{a^m} = a^{n-m}\] }{} 
    \command{ \[  \frac{a^n}{b^n} = \left(\frac{a}{b}\right)^n\] }{} 
    Siehe dazu auch: \url{https://de.wikipedia.org/wiki/Bruchrechnung}

\end{minipage} % End the first column of text
} % End the first division of the page
%----------------------------------------------------------------------------------------
%	SECOND COLUMN SPECIFICATION 
%----------------------------------------------------------------------------------------

\put(105,180){ % Divide the page
\begin{minipage}[t]{85mm} % Create a box to house text

%----------------------------------------------------------------------------------------
%	HEADING FOUR
%----------------------------------------------------------------------------------------
    \sectiontitle{Potenzen/Wurzeln}

    \command{ \[ a^{{-r}}={\frac  {1}{a^{r}}} \text{ \bf \ und \ } {\displaystyle a^{-{\frac {k}{n}}}={\frac {1}{a^{\frac {k}{n}}}} }\] }{}
    \command{ \[ a^{{{\frac  {m}{n}}}}={\sqrt[ {n}]{a^{m}}}=({\sqrt[ {n}]a})^{m} \] }{}
    \command{ \[ a^{{r-s}}={\frac  {a^{r}}{a^{s}}} \] }{}
    \command{ \[ (a\cdot b)^{r}=a^{r}\cdot b^{r} \text{\bf \ und \ }  {\sqrt[{n}]{a}}\cdot {\sqrt[{n}]{b}}={\sqrt[{n}]{a\cdot b}}  \] }{}
    \command{ \[ \left({\frac  {a}{b}}\right)^{r}={\frac  {a^{r}}{b^{r}}} \text{\bf \ und \ } {\frac {\sqrt[{n}]{a}}{\sqrt[{n}]{b}}}={\sqrt[{n}]{\frac {a}{b}}} \] }{}
    \command{ \[ (a^{r})^{s}=a^{{r\cdot s}} \text{\bf \ und \ } {\sqrt[ {m}]{{\sqrt[ {n}]{a}}}}={\sqrt[ {m\cdot n}]{a}} \] }{}
    \command{ \[ a^0 = 1\]}{}

    Siehe dazu auch: \url{https://de.wikipedia.org/wiki/Wurzel_(Mathematik)} sowie
    \url{https://de.wikipedia.org/wiki/Potenz_(Mathematik)}

\end{minipage} % End the second column of text
} % End the second division of the page

%----------------------------------------------------------------------------------------
%	THIRD COLUMN SPECIFICATION 
%----------------------------------------------------------------------------------------

\put(200,180){ % Divide the page
\begin{minipage}[t]{85mm} % Create a box to house tex

%----------------------------------------------------------------------------------------
%	IMPORTANT FILES
%----------------------------------------------------------------------------------------

    \sectiontitle{Binomische Formeln}
    \command{ \[ {\displaystyle (a+b)^{2}=a^{2}+2ab+b^{2}}\] }{}
    \command{ \[ {\displaystyle (a-b)^{2}=a^{2}-2ab+b^{2}}\] }{}
    \command{\[ (a+b)\cdot (a-b)=a^{2}-b^{2} \] }{}

    \sectiontitle{ \glqq{}Komische Identitäten\grqq{}}
    \command{ \[x = x + 0 \text{ \bf \ und \ } x = x \cdot 1  \] }{}
    \command{ \[ x = x + a -a \] }{}
    \command{ \[ x = \frac{\frac{1}{1}}{x} \] }{}
    \command{ \[ x = ln(e^x) \] }{}
    \command{ \[ f(x) = e^{ln(f(x))} \] }{}
%----------------------------------------------------------------------------------------
%	FOOTNOTE
%----------------------------------------------------------------------------------------

\vspace{\baselineskip}
\linethickness{0.5mm} % Thickness of the footer line
{\color{mygray}\line(1,0){30}} % Print the line with a custom color

\footnotesize{
Created by Aaron Wey, 2022. Last updated \today \\
% Todo: Auf Github verlinken	
Released under the MIT License.
}

%----------------------------------------------------------------------------------------

\end{minipage} % End the third column of text
} % End the third division of the page
\end{picture} % End the container for the entire page

%----------------------------------------------------------------------------------------

\end{document}
