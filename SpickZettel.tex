%%%%%%%%%%%%%%%%%%%%%%%%%%%%%%%%%%%%%%%%%
% Cheatsheet
% LaTeX Template
% Version 1.0 (12/12/15)
%
% This template has been downloaded from:
% http://www.LaTeXTemplates.com
%
% Original author:
% Michael Müller (https://github.com/cmichi/latex-template-collection) with
% extensive modifications by Vel (vel@LaTeXTemplates.com)
%
% License:
% The MIT License (see included LICENSE file)
%
%%%%%%%%%%%%%%%%%%%%%%%%%%%%%%%%%%%%%%%%%

%----------------------------------------------------------------------------------------
%	PACKAGES AND OTHER DOCUMENT CONFIGURATIONS
%----------------------------------------------------------------------------------------

\documentclass[11pt]{scrartcl} % 11pt font size

\usepackage[ngerman]{babel}
\usepackage[utf8]{inputenc} % Required for inputting international characters
\usepackage[T1]{fontenc} % Output font encoding for international characters

\usepackage[margin=0pt, landscape]{geometry} % Page margins and orientation

\usepackage{graphicx} % Required for including images

\usepackage{xcolor} % Required for color customization
\definecolor{mygray}{gray}{.75} % Custom color

\usepackage{url} % Required for the \url command to easily display URLs
\usepackage{tikz} % Required for the \keystroke command defined below to easily display Keys to press

\usepackage[ % This block contains information used to annotate the PDF
colorlinks=false, 
pdftitle={Cheatsheet}, 
pdfauthor={Aaron Wey, Markus Richter}, 
pdfsubject={Compilation of useful shortcuts}, 
pdfkeywords={Bash, Cheatsheet}
]{hyperref}

\setlength{\unitlength}{1mm} % Set the length that numerical units are measured in
\setlength{\parindent}{0pt} % Stop paragraph indentation

\renewcommand{\dots}{\ \dotfill{}\ } % Fills in the right amount of dots

\newcommand{\command}[2]{#1~\dotfill{}~#2\\} % Custom command for adding a shorcut

\newcommand{\sectiontitle}[1]{\paragraph{#1} \ \\} % Custom command for subsection titles


\colorlet{cmdColor}{blue}
\colorlet{argColor}{red}


\colorlet{cmdColor}{blue}
\colorlet{argColor}{red}

\newcommand{\Cmd}[1]{
\color{cmdColor}\$ #1
\color{black}
}

\newcommand{\Arg}[1]{
\color{argColor}#1 \color{black}
}


\newcommand{\Cmdp}[1]{
\color{cmdColor} #1
\color{black}
}
\usetikzlibrary{arrows, shapes, positioning, shadows}

\newcommand*\keystroke[1]{%
	\tikz[baseline=(key.base)]
	\node[%
	draw,
	fill=white,
	drop shadow={shadow xshift=0.25ex,shadow yshift=-0.25ex,fill=black,opacity=0.75},
	rectangle,
	rounded corners=2pt,
	inner sep=1pt,
	inner xsep=2pt,
	line width=0.5pt,
	font=\scriptsize\sffamily
	](key) {#1\strut}
	;
}

%----------------------------------------------------------------------------------------

\begin{document}

\begin{picture}(297,210) % Create a container for the page content

%----------------------------------------------------------------------------------------
%	TITLE SECTION 
%----------------------------------------------------------------------------------------

\put(10,200){ % Position on the page to put the title
\begin{minipage}[t]{210mm} % The size and alignment of the title
\section*{Übersicht der im Unix-Crashkurs behandelten Befehle} % Title
\end{minipage}
}

%----------------------------------------------------------------------------------------
%	FIRST COLUMN SPECIFICATION
%----------------------------------------------------------------------------------------

\put(10,180){ % Divide the page
\begin{minipage}[t]{85mm} % Create a box to house text


\sectiontitle{Nützliche Tasten}

\command{\keystroke{Tab}}{Autovervollständigung}
\command{\keystroke{Tab} \keystroke{Tab}}{Zeigt sinnvolle Argumente}
\command{\keystroke{Shift}+\keystroke{Tab}}{Iteriert über die Argumente rückwärts} %Todo: Passende Forulierung finden
\command{\keystroke{Pos1}, \keystroke{Strg}+\keystroke{a}}{Springt zum Anfang der Eingabezeile}
\command{\keystroke{Ende}, \keystroke{Strg}+\keystroke{e}}{Springt zum Ende der Eingabezeile}
\\
\command{\keystroke{$\uparrow$}, \keystroke{$\downarrow$}}{Iteriert durch letzte Eingaben}
\command{\keystroke{Strg}+\keystroke{r}}{Sucht in den letzten Befehlen}
\command{\keystroke{Strg}+\keystroke{c}}{Bricht den aktuellen Befehl ab}

\sectiontitle{Hilfsbefehle}

\command{\Cmd{man} \Arg{Befehl}}{Öffnet die \glqq  Manual-Page\grqq{}}
\command{\Cmd{Befehl} \Arg{-h}}{Öffnet die Hilfe zu diesem Befehl}
\command{\Cmd{apropos} \Arg{Suchwort}}{Durchsucht die Man-Pages}
\command{\Cmd{whatis} \Arg{Befehl}}{Zeigt Kurzbeschreibung}
			
\sectiontitle{Orientierung}

%\command{\Cmd{pwd}}{Zeige den momentanen Pfad}
\command{\Cmd{ls}}{Zeige den Ordnerinhalt}
\command{\Cmd{cd} \Arg{Zielpfad}}{Wechsle nach Zielpfad}
\command{\Cmd{exit}}{Beende die aktuelle Kommandozeile}

\sectiontitle{Dateibefehle}

\command{\Cmd{mv} \Arg{Quellpfad Zielpfad}}{Verschiebt Ziel}
\command{\Cmd{cp} \Arg{Quellpfad Zielpfad}}{Kopiert Ziel}
\command{\Cmd{rm} \Arg{Dateipfad}}{Löscht Datei/Ordner}
\command{\Cmd{mkdir} \Arg{Zielpfad}}{Erstellt einen neuen Ordner }
\command{\Cmd{cat} \Arg{Zielpfade}}{Gibt eine oder mehrere Dateien nacheinander aus}
\command{\Cmd{less} \Arg{Dateipfad}}{Zeigt die Datei an}
\command{\Cmd{grep} \Arg{SuchWort Dateipfad}}{Gibt alle Zeilen mit SuchWort aus}




\end{minipage} % End the first column of text
} % End the first division of the page
%----------------------------------------------------------------------------------------
%	SECOND COLUMN SPECIFICATION 
%----------------------------------------------------------------------------------------

\put(105,180){ % Divide the page
\begin{minipage}[t]{85mm} % Create a box to house text

%----------------------------------------------------------------------------------------
%	HEADING FOUR
%----------------------------------------------------------------------------------------
\sectiontitle{Editoren}

\command{\Cmd{nano} \Arg{Dateipfad}}{Öffnet Datei in Nano}
\command{\keystroke{Strg}+\keystroke{x}}{Nano schließen}
\command{\Cmd{vim} \Arg{Dateipfad}}{Öffnet Datei in Vim}
\command{\keystroke{Esc}\keystroke{:}\keystroke{w}\keystroke{q}\keystroke{Enter}}{Vim mit Speichern schließen}
\command{\keystroke{Esc}\keystroke{:}\keystroke{q}\keystroke{!}\keystroke{Enter}}{Vim ohne Speichern schließen}


					
%----------------------------------------------------------------------------------------
%	HEADING FIVE
%----------------------------------------------------------------------------------------				
					
\sectiontitle{Benutzer und Rechte} % Heading five

\command{\Cmd{ls} \Arg{-la Pfad}}{Zeige den Ordnerinhalt(inkl. versteckten Dateien) von Pfad als Tabelle}%TODO kurze Rechteauflistung (Owner/Group/All)
\command{\Cmd{sudo} \Arg{Befehl}}{Führe Befehl als root-Nutzer aus}
\command{\Cmd{chmod} \Arg{Änderungen Dateipfad(e)}}{Ändere Rechte}
\command{\Cmd{chown} \Arg{Nutzer:Gruppe Dateipfad(e)}}{Andere Eigentümer}

\sectiontitle{Pipes und Befehlsketten}

\tikzstyle{cmdBlock} = [rectangle, draw, fill=blue!20, text width=3em, text centered, rounded corners, minimum height=7em, inner sep=0pt]
\tikzstyle{cmdOut} = [isosceles triangle, draw, minimum width=0.1cm, text centered, minimum height=1.3cm, inner sep=0pt]

\command{\Cmd{Befehl1} \Arg{Argumente} \&\& \Cmdp{Befehl2} \Arg{Argumente}}{Führe erst Befehl1, danach Befehl2 aus, falls Befehl1 keine Fehler hatte}
\command{\Cmd{Befehl1} \Arg{Argumente} ; \Cmdp{Befehl2} \Arg{Argumente}}{Führe erst Befehl1, danach auf jeden Fall Befehl2 aus}
\command{\Cmd{Befehl1} \Arg{Argumente} | \Cmdp{Befehl2} \Arg{Argumente}}{Leite die Ausgabe von Befehl1 als Eingabe in Befehl2 um}
\command{\Cmd{Befehl} < \Arg{Datei}}{Inhalt der Datei als Befehlsinput}
\command{\Cmd{Befehl} > \Arg{Datei}}{Speichert Ausgabe in Datei (nur Fehlerausgabe: 2>)}
\command{\Cmd{Befehl} {>}{>} \Arg{Datei}}{Schreibt Ausgabe an das Ende der Datei}
\command{\Cmd{Befehl} 2>\&1 > \Arg{Datei}}{Schreibt alle Ausgaben in Datei}

\end{minipage} % End the second column of text
} % End the second division of the page

%----------------------------------------------------------------------------------------
%	THIRD COLUMN SPECIFICATION 
%----------------------------------------------------------------------------------------

\put(200,180){ % Divide the page
\begin{minipage}[t]{85mm} % Create a box to house tex

%----------------------------------------------------------------------------------------
%	IMPORTANT FILES
%----------------------------------------------------------------------------------------

\sectiontitle{Globbing (Mehrere Pfade auf einmal angeben)}
\command{\Arg{*}}{Alle nicht-versteckten Dateien und Ordner}
\command{\Arg{*/}}{Alle nicht-versteckten Ordner}
\command{\Arg{*a*}}{Alle nicht-versteckten Dateien und Ordner, die a enthalten}
\command{\Arg{?}}{irgend ein einzelnes Zeichen}
\command{\Arg{\{a,b,c\}.pdf}}{ist a.pdf,b.pdf und c.pdf}


\sectiontitle{Wichtige Programme}
\command{\Cmd{ssh} \Arg{Benutzer@Ziel-Domain}}{Öffnet SSH-Verbindung}
\command{\Cmd{rsync} \Arg{Datei Benutzer@Ziel-Domain:Zielpfad}}{Kopiert Datei auf entfernten Server}
%\sectiontitle{Wichtige interaktive Programme}
\command{\Cmd{tar} \Arg{-cf Archiv.tar Quellpfade}}{Erstelle Archiv}
\command{\Cmd{tar} \Arg{-czf Archiv.tar.gz Quellpfade}}{Erstelle komprimiertes Archiv}
\command{\Cmd{tar} \Arg{-xf Archivdatei}}{Entpacke Archiv}
\command{\Cmd{top}}{Zeige Prozesse und Auslastung des Computers (besser: \Cmd{htop})}
\command{\Cmd{killall} \Arg{Prozessname}}{beende laufenden Prozess (auch: \Cmd{pkill})}



%----------------------------------------------------------------------------------------
%	LINKS AND INFORMATION
%----------------------------------------------------------------------------------------


%----------------------------------------------------------------------------------------
%	FOOTNOTE
%----------------------------------------------------------------------------------------

\vspace{\baselineskip}
\linethickness{0.5mm} % Thickness of the footer line
{\color{mygray}\line(1,0){30}} % Print the line with a custom color

\footnotesize{
Created by Aaron Wey, Markus Richter, 2019. Last updated 2020\\ 
\url{https://hu.berlin/unixkurs20}\\
				
Released under the CC-BY-SA license.
}

%----------------------------------------------------------------------------------------

\end{minipage} % End the third column of text
} % End the third division of the page
\end{picture} % End the container for the entire page

%----------------------------------------------------------------------------------------

\end{document}
